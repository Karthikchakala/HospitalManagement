% Chapter Template

\chapter{Implementation: Patient Workflows} % Main chapter title

\label{Chapter4} % Change X to a consecutive number; for referencing this chapter elsewhere, use \ref{Chapter4}

\lhead{Chapter 4. \emph{Implementation: Patient Workflows}} % Change X to a consecutive number; this is for the header on each page - perhaps a shortened title

%----------------------------------------------------------------------------------------
%	SECTION 1
%----------------------------------------------------------------------------------------

\section{Patient-Facing Implementation Overview}

Implementation overview: We implemented patient-facing workflows including appointment discovery, booking, and management. The patient dashboard exposes quick actions, including a “View Appointments” card linking to upcoming bookings. The appointments page fetches scheduled visits, supports cancel/reschedule, and hides the booking form in view-only mode using a `mode=view` query parameter. Email notifications are sent on booking and changes.

%-----------------------------------
%	SUBSECTION 1
%-----------------------------------
\subsection{Frontend Flows}
Frontend flows: Built with Next.js and Tailwind CSS. Reusable components manage loading states and accessibility. Client code calls backend APIs via Axios, using \texttt{NEXT\_PUBLIC\_BACKEND\_BASE\_URL}. State management relies on React hooks; UI provides responsive cards, tables, and dialogs for cancel/reschedule with clear feedback.

%-----------------------------------
%	SUBSECTION 2
%-----------------------------------

\subsection{Backend Services}
Backend services: Express routes under `/api/patient/appointments/*` unify read and write flows, with JWT middleware enforcing access. Nodemailer sends confirmations and changes; cancellation and rescheduling trigger appropriate updates and emails. Virtual appointments include WebRTC links built from \texttt{FRONTEND\_BASE}, ensuring correct meeting URLs.

%----------------------------------------------------------------------------------------
%	SECTION 2
%----------------------------------------------------------------------------------------

\section{Security and Reliability}

Security and reliability: Requests use HTTPS and JWT; sensitive IDs are server-resolved where possible. Error handling returns actionable messages and preserves invariant checks (ownership, valid status transitions). Logging enables traceability without exposing PHI, and rate-limiting is applied to sensitive endpoints as needed.