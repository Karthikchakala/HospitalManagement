% Chapter Template

\chapter{Related Work and Requirements} % Main chapter title

\label{Chapter2} % Change X to a consecutive number; for referencing this chapter elsewhere, use \ref{Chapter2}

\lhead{Chapter 2. \emph{Related Work and Requirements}} % Change X to a consecutive number; this is for the header on each page - perhaps a shortened title

%----------------------------------------------------------------------------------------
%	SECTION 1
%----------------------------------------------------------------------------------------

\section{Related Systems and Literature}

This chapter surveys related systems and positions our Hospital Management System (HMS) within the broader landscape of clinical information systems. We examine appointment schedulers, EMR/EHR platforms, and telemedicine tools, highlighting the fragmentation that occurs when institutions combine disparate point solutions. Our HMS integrates these capabilities end-to-end to minimize context switching, reduce data duplication, and improve care coordination.

%-----------------------------------
%	SUBSECTION 1
%-----------------------------------
\subsection{Standards and Interoperability}
Open-source HMS options provide modular building blocks but often lack real-time communication and analytics. Enterprise suites deliver breadth but are costly and rigid. Standards such as HL7/FHIR inform data exchange, yet practical interoperability remains challenging for mid-sized hospitals. Our approach balances pragmatism and standards alignment—Supabase (Postgres) for structured data, REST APIs for service boundaries, and exportable reports for integration.

%-----------------------------------
%	SUBSECTION 2
%-----------------------------------

\subsection{Key Differentiators}
Key differentiators of our HMS include unified appointment management (in-person and virtual via WebRTC), role-based workspaces for patients, doctors, staff, and admins, and a first-class analytics experience using Recharts. We emphasize maintainable TypeScript codebases on both frontend and backend, auditable actions, and secure authentication with JWT.

%----------------------------------------------------------------------------------------
%	SECTION 2
%----------------------------------------------------------------------------------------

\section{Functional and Non-Functional Requirements}
Requirements distilled from the survey include: centralized appointment scheduling, integrated EMR and diagnostics, streamlined billing with online payments, secure communication (chat and video), and actionable admin analytics. Non-functional goals prioritize security, privacy, reliability, scalability, and usability suited to clinic and hospital settings.